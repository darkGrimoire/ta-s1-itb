\chapter{Studi Literatur}

\section{Pembelajaran}
Belajar adalah perubahan yang relatif permanen dalam perilaku atau potensi perilaku sebagai hasil dari pengalaman atau latihan yang diperkuat \parencite{slavin2017}. Belajar merupakan akibat adanya interaksi antara stimulus dan respons. Seseorang dianggap telah belajar sesuatu jika dia dapat menunjukkan perubahan perilakunya. Menurut teori ini, dalam belajar yang penting adalah input yang berupa stimulus dan output yang berupa respons.

Salah satu pengertian pembelajararan dikemukakan oleh \textcite{gagne1970} yaitu pembelajaran adalah seperangkat peristiwa -peristiwa eksternal yang dirancang untuk mendukung beberapa proses belajar yang bersifat internal. Lebih lanjut, Gagne (1985) mengemukakan teorinya lebih lengkap dengan mengatakan bahwa pembelajaran dimaksudkan untuk menghasilkan belajar, situasi eksternal harus dirancang sedemikian rupa untuk mengaktifkan, mendukung, dan mempertahankan proses internal yang terdapat dalam setiap peristiwa belajar.

\section{Pemrograman}
\textcite{hartree2012calculating} menjelaskan bahwa "Proses mempersiapkan kalkulasi untuk mesin dapat dibagi menjadi 2 bagian, 'pemrograman' dan 'pengkodean'. Pemrograman adalah proses menggambarkan penjadwalan dari urutan operasi-operasi individu yang dibutuhkan untuk melakukan kalkulasi" \parencite[p.~111]{hartree2012calculating}.

\section{Pembelajaran Pemrograman secara Daring}
\blindtext

\section{\textit{Interactive Coding}}
\blindtext

\section{Arsitektur Infrastruktur}
\blindtext

\section{Kontainer Eksekutor}
\blindtext

\section{Orkestrasi Kontainer}
\blindtext

\section{Komunikasi Frontend-Backend}
\blindtext

