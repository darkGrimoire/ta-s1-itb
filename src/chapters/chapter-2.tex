\chapter{Studi Literatur}

\section{Pembelajaran}
Menurut \textcite{slavin2017learn}, belajar adalah perubahan yang relatif permanen dalam perilaku atau potensi perilaku sebagai hasil dari pengalaman atau latihan yang diperkuat. Belajar merupakan akibat adanya interaksi antara stimulus dan respons. Seseorang dianggap telah belajar sesuatu jika dia dapat menunjukkan perubahan perilakunya. Menurut teori ini, dalam belajar yang penting adalah input yang berupa stimulus dan output yang berupa respons.

Teori lain mengenai pembelajaran juga dikemukakan oleh \textcite{gagne1970learning} bahwa pembelajaran adalah seperangkat peristiwa-peristiwa eksternal yang dirancang untuk mendukung beberapa proses belajar yang bersifat internal. Pembelajaran terdiri dari beberapa tipe dan setiap tipe membutuhkan instruksi yang berbeda-beda. Menurut \textcite{gagne1970learning}, terdapat 5 kategori utama dalam pembelajaran: informasi verbal, kemampuan intelektual, strategi kognitif, kemampuan motorik, serta sikap. Setiap kategori memiliki kondisi internal dan eksternal masing-masing dalam proses pembelajarannya.

Menurut \textcite{gagne1985learning}, terdapat 9 peristiwa instruksi, yaitu: Perhatian dan Motivasi, Pemberitahuan Objektif Pembelajaran, Stimulasi Pengulangan, Pemberian Tantangan (Stimulus), Memberikan Arahan, Keterlibatan Langsung/Pengalaman, Memberikan Saran, Penilaian Performa, serta Meningkatkan Daya Ingatan. Dengan memanfaatkan peristiwa-peristiwa instruksi tersebut, instruktor dapat mempercepat pembelajaran internal pelajar. Hal-hal ini dapat dipakai dalam teknik pembelajaran sehingga hasil pembelajaran menjadi lebih efektif dan bertahan lama.

\section{Pemrograman}
Pada masa awal publikasi dalam pemrograman, banyak praktisi yang menjelaskan pemrograman sebagai proses yang menerjemahkan bahasa yang diketahui manusia menjadi bahasa yang dimengerti oleh mesin \parencite{mccracken1957digital}. \textcite{booth1958programming} menambahkan hubungan antara pemrograman dengan kalkulasi bahwa "Proses mengorganisasi kalkulasi dapat dibagi menjadi dua bagian -- fondasi formula matematis dan pemrograman yang sebenarnya ... menerjemahkan ... ke dalam bahasa mesin komputasi". \textcite{hartree2012calculating} menjelaskan bahwa "Proses mempersiapkan kalkulasi untuk mesin dapat dibagi menjadi 2 bagian, 'pemrograman' dan 'pengkodean'. Pemrograman adalah proses menggambarkan penjadwalan dari urutan operasi-operasi individu yang dibutuhkan untuk melakukan kalkulasi" \parencite{hartree2012calculating}.

Definisi pemrograman berubah seiring waktu dari domain matematika menjadi domain pemrosesan data yang lebih umum \parencite{hoc1990psychology}. Pada awalnya, pemrogram adalah orang yang serius dalam dunia komputer. Namun seiring dengan waktu, aplikasi-aplikasi besar banyak yang mulai menerapkan pemrograman untuk \textit{scripting} atau \textit{macro} sehingga definisi pemrograman dapat meluas hingga ke \textit{end-user} \parencite{goodell1999enduser}. Hal ini membuat \textcite{blackwell2002programming} mengusulkan bahwa pemrogram bisa menjadi siapapun yang menggunakan komputer dengan penggunaan yang berbeda-beda tergantung pada kegunaan dan kebutuhan.

\section{Pembelajaran Pemrograman secara Daring}
\subsection{Perkembangan Pembelajaran secara Daring}
Pembelajaran secara daring biasa digunakan untuk merujuk kepada pembelajaran berbasis web, pembelajaran terdistribusi, pembelajaran berbasis internet, pembelajaran siber, atau pembelajaran virtual \parencite{urdan2000elearning}. Pembelajaran secara daring adalah bagian dari pembelajaran jarak jauh yang memeluk berbagai macam aplikasi teknologi dan proses pembelajaran, termasuk pembelajaran berbasis komputer, pembelajaran berbasis web, kelas virtual, dan kolaborasi digital \parencite{urdan2000elearning}. Hal ini menunjukkan bahwa pembelajaran secara daring memiliki banyak sekali media yang dapat digunakan, mulai dari internet, siaran satelit, rekaman audio/video, CD, dll.

Pembelajaran secara daring bukanlah hal yang baru, namun dengan perkembangan teknologi zaman ini media pembelajaran secara daring semakin bervariasi dan kompleks. Berikut adalah tabel yang memberikan penjelasan lebih lanjut terkait perkembangan dalam pembelajaran secara daring dalam 30 tahun terakhir \parencite{herrington2005online,mortera2006faculty,nicholson2005elearning,pilla2006characterising,keengwe2010towards}.

\begin{longtable}{ |p{0.15\linewidth}|p{0.3\linewidth}|p{0.45\linewidth}| }
  \caption{\label{tab:sejarah-pembelajaran-daring}Sejarah Perkembangan Pembelajaran secara Daring}                                                                                                                                                                                                                                                             \\
  \hline
  \rowcolor{gray!30}
  Tahun         & Fokus                                                                & Karakteristik Sistem Edukasi                                                                                                                                                                                                                                          \\
  \hline
  1975-1985     & Pemrograman; Latihan Soal; \textit{Computer-assisted learning (CAL)} & Pendekatan pembelajaran secara behavioristik; Pemrograman untuk membangun alat dan pemecahan masalah; Interaksi pengguna-komputer secara lokal.                                                                                                                       \\
  \hline
  1983-1990     & Pelatihan multimedia berbasis komputer                               & Penggunaan model CAL terdahulu dengan integrasi multimedia interaktif; Dominansi model pelajar pasif; Influensi dari pembelajaran menggunakan alat peraga mulai terlihat.                                                                                             \\
  \hline
  1990-1995     & Pelatihan dan edukasi berbasis web                                   & Penyampaian materi melalui internet; Pengembangan model pelajar aktif; Pembelajaran dengan alat peraga menjadi umum; Interaksi pengguna secara terbatas.                                                                                                              \\
  \hline
  1995-2005     & \textit{E-learning}                                                  & Penyampaian materi melalui internet dengan lebih fleksibel; Peningkatan interaktivitas; Penggunaan kelas multimedia secara daring; Interaksi pengguna jarak jauh.                                                                                                     \\
  \hline
  2005-sekarang & Pembelajaran melalui gawai dan jaringan sosial                       & Adanya materi pembelajaran interaktif melalui \textit{Learning Management System} (LMS) dengan komponen jaringan sosial; Pembelajaran yang difasilitasi gawai nirkabel seperti ponsel dan laptop; Pembelajaran secara portabel dengan fokus kepada mobilitas pelajar. \\
  \hline
\end{longtable}

\subsection{Pembelajaran Pemrograman Interaktif}
Pembelajaran pemrograman secara daring saat ini menggunakan berbagai macam media dalam penyampaian materinya. Mulai dari berbentuk teks, video, animasi, \textit{mentoring}, \textit{bootcamp}, \textit{webinar}, latihan soal, hingga simulasi berbentuk gamifikasi seperti pada \textcite{codingame2021media}. Seiring dengan bertambahnya kompetitor, beragam situs mulai mengadopsi sistem pembelajaran yang lebih interaktif, mudah dipahami, dan menyenangkan agar membedakan dengan kompetitor lainnya dan menjadi nilai tambah bagi penggunanya.

\begin{figure}[H]
  \centering
  \includegraphics[width=0.7\textwidth]{chapter2/codingame.png}
  \caption{Gamifikasi pada \href{https://www.codingame.com}{CodinGame}}
\end{figure}

\textcite{sololearn2021media} menggunakan materi berbentuk teks yang disertai latihan soal pemrograman berbentuk pilihan ganda, isian singkat, serta \textit{drag \& drop} jawaban pada tempat yang telah disediakan. \href{https://www.sololearn.com}{Sololearn} juga memiliki editor kode daring yaitu tempat pengguna dapat mencoba langsung membuat dan menjalankan kode melalui situsnya tanpa perlu melakukan instalasi apapun. \href{https://www.sololearn.com}{Sololearn} berfokus terhadap latihan implementasi program secara langsung dan mempelajari berbagai macam fitur dan logika dalam bahasa tersebut.

\begin{figure}[H]
  \centering
  \includegraphics[width=0.7\textwidth]{chapter2/sololearn.png}
  \caption{Belajar dengan latihan soal pada \href{https://www.sololearn.com}{Sololearn}}
\end{figure}

Sementara itu, \href{https://brilliant.org}{Brilliant} menggunakan teknik yang berbeda. \textcite{brilliant2021media} memakai eksekusi pemrograman yang lebih visual dan interaktif sehingga membuat pembelajaran lebih menarik dan menyenangkan. \href{https://brilliant.org}{Brilliant} juga menggunakan berbagai macam visual pendukung seperti animasi dan grafis yang menarik dan mudah dipahami, serta juga menggunakan bahasa yang dinarasikan sedemikian rupa sehingga konsep pemrograman yang rumit menjadi mudah dipahami.

\begin{figure}[H]
  \centering
  \includegraphics[width=0.3\textwidth]{chapter2/brilliant.png}
  \caption{Salah satu pembelajaran interaktif pada \href{https://brilliant.org}{Brilliant}}
\end{figure}

Selain dari \href{https://www.sololearn.com}{Sololearn} dan \href{https://brilliant.org}{Brilliant}, terdapat juga platform-platform lainnya seperti \textcite{katacoda2021media} yang menyediakan kelas pemrograman interaktif menggunakan web editor/terminal dan instruksi dalam bentuk teks sehingga pengguna langsung dapat terjun mempelajari dan mencoba materi yang diberikan. Katacoda juga menyediakan seluruh kelasnya secara gratis untuk siapapun. Pendekatan seperti Katacoda juga dapat kita lihat pada platform lain seperti \textcite{codesaya2021media} yang memanfaatkan web editor dan memberikan feedback berupa hasil eksekusi program dalam bentuk teks.

\begin{figure}[H]
  \centering
  \includegraphics[width=0.7\textwidth]{chapter2/katacoda.png}
  \caption{Terminal interaktif pada kelas Docker di \href{https://www.katacoda.com/}{Katacoda}}
\end{figure}

\begin{figure}[H]
  \centering
  \includegraphics[width=0.7\textwidth]{chapter2/codesaya.png}
  \caption{Contoh konten kelas pada \href{https://www.codesaya.com/}{CodeSaya}}
\end{figure}

Mirip dengan \href{https://www.katacoda.com/}{Katacoda} yang menggunakan feedback berupa hasil eksekusi kode namun masih menggunakan web editor juga, terdapat juga \textcite{froggy2021media} beserta permainan lainnya dari \href{https://codepip.com/games/}{Codepip} yang berfokus pada pembelajaran spesifik dalam pembangunan web. \href{https://flexboxfroggy.com/}{Flexbox Froggy} menggunakan feedback berupa visual sehingga hasil \textit{style CSS} yang dibuat dapat langsung terlihat oleh pengguna dan disesuaikan agar dapat memenuhi tujuan yang diinginkan selama mengerjakan latihan.

\begin{figure}[H]
  \centering
  \includegraphics[width=0.7\textwidth]{chapter2/froggy.png}
  \caption{Visualisasi penggunaan flexbox di \href{https://www.flexboxfroggy.com/}{Flexbox Froggy}}
\end{figure}

Terdapat banyak platform pembelajaran pemrograman secara daring yang mengintegrasikan permainan ke dalam metode pembelajarannya, baik itu melalui latihan praktiknya, penjelasan materinya, hingga sistem pengembangan diri pada platform itu sendiri. Salah satu platform yang memiliki ketiganya adalah \textcite{progate2021media} yang menarasikan materi menggunakan maskot serta cerita yang mudah dipahami, latihan praktik yang langsung terjun implementasi, serta sistem level dan \textit{achievement/badges} bagi pengguna yang aktif menggunakan platformnya.

\begin{figure}[H]
  \centering
  \includegraphics[width=0.7\textwidth]{chapter2/progate-intro.png}
  \caption{Penyampaian materi pada \href{https://www.progate.com/}{Progate}}
\end{figure}

\begin{figure}[H]
  \centering
  \includegraphics[width=0.7\textwidth]{chapter2/progate-code.png}
  \caption{Latihan implementasi pada \href{https://www.progate.com/}{Progate}}
\end{figure}

Salah satu studi terkait pembelajaran interaktif terminal UNIX uAssign \parencite{bailey2019uassign} memberikan konklusi bahwa pembelajaran interaktif menggunakan uAssign menghasilkan peningkatan di kemampuan menggunakan terminal siswa yang menggunakannya, terutama bagi siswa yang belum mengetahui  penggunaan terminal. Studi lain pada \href{https://pythontutor.com}{Online Python Tutor} \parencite{guo2013pythontutor}---aplikasi web yang memberikan visualisasi eksekusi kode Python---visualisasi yang disediakan oleh Online Python Tutor sangat membantu dalam proses pembelajaran. Eksekusinya yang dilakukan secara daring juga memudahkan siswa dalam menggunakan aplikasi tersebut pada forum pembelajaran untuk membantu siswa lain dalam memecahkan permasalahan, tanpa harus melakukan instalasi ataupun konfigurasi terlebih dahulu.

