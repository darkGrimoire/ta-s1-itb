\clearpage
\chapter*{ABSTRAK}
\addcontentsline{toc}{chapter}{ABSTRAK}

%taruh abstrak bahasa indonesia di sini
\begin{center}
  \center
  \large \bfseries \MakeUppercase{\thetitle}

  \normalfont\normalsize
  Oleh

  \theauthor
\end{center}

\vspace{1cm}

\begin{singlespace}
  Abstrak berisi ringkasan apa yang telah dikerjakan dalam tugas akhir. Ada beberapa hal yang perlu diperhatikan dalam penulisan abstrak. Pertama, abstrak harus memuat permasalahan yang dikaji, metode/teknik yang digunakan untuk menyelesaikan masalah, hasil yang dicapai / evaluasi kajian, kesimpulan yang diperoleh, dan kata kunci.  Kedua, cara penulisannya harus padat dan terarah. Setiap kalimat harus dapat memberikan informasi sebanyak dan setepat mungkin, mudah dibaca dan dimengerti. Panjang ringkasan dibatasi maksimal 300 kata dan ditulis dengan satu spasi. Panjang ringkasan dibatasi maksimal 300 kata dan ditulis dengan satu spasi.
\end{singlespace}

\textbf{\textit{Kata kunci: ringkasan, singkat, padat.}}
\clearpage