\clearpage
\chapter*{ABSTRACT}
\addcontentsline{toc}{chapter}{ABSTRACT}

%taruh abstrak bahasa indonesia di sini
\begin{center}
  \center
  \begin{singlespace}
    \bfseries \MakeUppercase{Development of Interactive Learning Environment (ILE) on Programming Learning Platform KodeBareng}

    \normalfont\normalsize
    By

    \bfseries \theauthor
  \end{singlespace}
\end{center}

% \vspace{1cm}

\begin{singlespace}
  One of the challenges in learning programming for people who are new to programming is understanding how programs work. This is important because they still don't have an appropriate image of the program's work process so they have difficulty absorbing the programming material learned. This is complicated when learning is done online, because learning can take place asynchronously so that teachers must be able to ensure understanding can be achieved through learning materials without direct interaction.

  One way to improve understanding in learning is by creating an interactive learning system (\textit{Interactive Learning Environment [ILE]}). In some existing online programming learning platforms, interactive learning is done by working on problems using a Web IDE such as Sololearn, understanding concepts using interactive visual animations such as Brilliant, to delivering material using interactive narration on Progate. However, the use of visualization of program execution as an ILE is still minimally used in programming learning classes, even though according to several literature studies it can improve understanding of programming concepts because it can generate concrete models of computer programming in ordinary students.

  In this Final Project, an ILE is built in the form of a tool that can visualize code execution that displays how the program can work. This system is integrated with online programming classes on the KodeBareng Web Platform so that it can be used in the learning process and practice working on problems. After conducting a user experiment (\textit{user experiment}), it was found that this ILE was considered by users to be helpful in learning with an average score in the range of 4.231 and 4.385 on the \textit{likert} scale, as well as increasing the average correct answer value on the questions tested with the most significant increase in concept understanding (using SOLO Level) in module 1 with a t value of 2.179 (\textit{p = 0.0147}) for people who have never learned programming before.

  \textbf{\textit{Keywords: programming learning, interactive learning, code execution visualization}}
\end{singlespace}
\clearpage
