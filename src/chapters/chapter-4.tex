\chapter{Implementasi dan Pengujian}
Rancangan Perangkat Lunak
Tujuan penulisan bab ini adalah untuk menunjukkan seberapa jauh solusi yang diuraikan pada bagian sebelumnya dapat menyelesaikan permasalahan utama Tugas Akhir. Metode yang dipakai untuk melakukan evaluasi dapat bermacam-macam, bergantung pada jenis permasalahannya.

\section{Implementasi}

 [\hl{TODO: JELASIN KOMPONEN DIAGRAM DISINI}]

\subsection{Batasan Implementasi}
\blindtext

\subsection{Implementasi Eksekutor}
\subsubsection{Implementasi Modul Eksekutor}
\blindtext
\subsubsection{Implementasi Modul Parser}
\blindtext
\subsubsection{Implementasi Modul Controller}
\blindtext

\subsection{Implementasi Backend}
\subsubsection{Implementasi Modul GradeHelper}
\blindtext


\subsection{Implementasi Frontend}
\subsubsection{Implementasi Modul WebIDE}
\blindtext
\subsubsection{Implementasi Modul Visualizer}
\blindtext


\section{Pengujian}

Liat di TA Source Code Analyzer sama yg Arduino

\subsection{Tujuan Pengujian}
\blindtext

\subsection{Lingkungan Pengujian}
\blindtext

\subsection{Skenario dan Hasil Pengujian}
\blindtext


\section{Analisis Hasil Pengujian}
\blindtext