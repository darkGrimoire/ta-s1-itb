\chapter{Analisis dan Perancangan}
Pada bab ini dituliskan analisis terkait kebutuhan sistem pada KodeBareng, masalah pada implementasi ILE yang sudah ada, serta rancangan solusi yang akan diimplementasikan

\section{Analisis Masalah}
%  [\hl{TODO: ANALISIS PERMASALAHAN DIKAITKAN DENGAN PERMASALAHAN PADA PEMROGRAMAN PEMULA. JELASIN JUGA KENAPA BUTUH PEMBELAJARAN INTERKATIF}]

%  [\hl{TODO: JELASIN JUGA PERMASALAHAN PADA PLATFORM2 YANG ADA, BAIK KELAS PEMROGRAMAN MAUPUN PYTHONTUTOR DSB}]

Kompleksitas dari pemrograman itu sendiri menyebabkan sulitnya pelajar yang baru mempelajari pemrograman memahami konsep-konsep dan abstrak dari pemrograman \parencite{moons2013pilot}. Kurangnya kemampuan dalam menelusuri jejak suatu kode program, serta belum terbentuknya model kerangka pikir terhadadp cara program komputer bekerja \parencite{mayer1981psychology} merupakan faktor-faktor yang membuat pelajar tidak dapat menyerap informasi teknis baru terkait pemrograman dengan maksimal. Maka dari itu, diperlukan suatu bentuk model konkret terkait alur proses kerja suatu program komputer sehingga dapat mempermudah proses transfer ilmu konsep-konsep pemrograman kepada pelajar. Pembelajaran interaktif menjadi salah satu cara yang dapat dimanfaatkan karena dapat diaplikasikan secara daring menggunakan teknologi digital yang sudah tersedia.

Menurut \textcite{moons2013pilot}, terdapat 4 pendekatan yang dapat dilakukan untuk memecahkan masalah tersebut. Pendekatan pertama adalah dengan menggunakan urutan paradigma pemrograman tertentu dalam pembelajarannya sehingga konsep-konsep yang lebih dalam pada pemrograman tidak harus dipelajari dari awal. Pendekatan kedua adalah dengan menggunakan teknik pembelajaran secara aktif, seperti lokakarya dan penceritaan dengan narasi. Namun, pendekatan ini agak sulit diaplikasikan pada pembelajaran secara daring. Pendekatan ketiga adalah dengan menggunakan bahasa pemrograman yang lebih sesuai untuk pelajar pemrograman, seperti Python dan Eiffel. Pendekatan keempat adalah menggunakan lingkungan pembelajaran interaktif atau \textit{interactive learning environment} (ILE) yang dapat berbentuk \textit{microworld}, visualisasi algoritma, dan visualisasi eksekusi program.

% KodeBareng adalah platform berbasis web sebagai sarana pembelajaran pemrograman menggunakan gamifikasi. Pada platform ini, dibutuhkan suatu \textit{interactive learning environment} (ILE) yang dapat mendukung aktivitas pemrograman secara praktis. ILE diharapkan dapat meningkatkan kemampuan pemecahan masalah dan implementasi kode penggunanya dengan adanya wadah menulis dan menjalankan kode secara daring. ILE dibangun secara modular pada platform KodeBareng sehingga tidak terlalu disruptif terhadap sistem yang sudah ada.

% Dalam beberapa sistem yang sudah ada seperti \href{https://olympia.id/}{Olympia}, ILE yang dapat menjawab kebutuhan ini biasanya berupa sistem submisi kode implementasi yang menggunakan autograder untuk penilaiannya. Sayangnya, hal ini dinilai kurang interaktif karena setiap perubahan kode yang dilakukan pengguna harus melakukan submisi ulang dan menjalankan kembali programnya. Pengguna juga menjadi tidak tahu kesalahannya dimana, serta masih harus dapat menjalankan kodenya secara manual di mesinnya.

% Pada beberapa platform lain seperti yang sudah dibahas pada bab sebelumnya, terdapat berbagai macam jenis ILE yang masing-masing memiliki kelebihan dan kekurangannya masing-masing. ILE visual seperti yang digunakan pada \textcite{brilliant2021media} memiliki interaktivitas yang sangat tinggi dan mudah dipahami, namun kurang dapat diimplementasikan secara umum dan tidak langsung menyentuh implementasi kode. Sementara itu, terdapat juga ILE yang berupa Web IDE namun kebanyakan implementasinya hanya sebatas eksekusi kode pada wadah teks yang diberikan. Keterbatasan ini membuat proses pembelajaran terhambat akibat kurangnya ada pembantu yang dapat digunakan oleh pengguna untuk mencari letak kesalahan dalam implementasi kodenya, serta menjadikan Web IDE terlalu menantang bagi pemula yang ingin belajar pemrograman.

Berdasarkan studi literatur terhadap beberapa platform kelas pembelajaran lainnya, metode pembelajaran pemrograman interaktif secara daring dapat dikategorikan menjadi visual dan non-visual. Metode pembelajaran pemrograman yang visual tidak langsung menggunakan kode seperti pada \textcite{brilliant2021media}, tetapi menggunakan perumpaan visual yang interaktif dengan memakai gambar, simbol, serta animasi. Metode ini membuat pembelajaran menjadi lebih menarik dan mudah, karena diekspresikan dalam bentuk visual sehingga dapat membentuk model kerangka pikir pelajar. Namun, metode ini tidak menyentuh secara langsung aktivitas pemrograman secara praktis sehingga ada kemungkinan dapat terjadi perbedaan antara teori dan praktik yang dipahami pelajar dengan implementasi program yang sebenarnya. Metode ini juga memiliki bentuk yang spesifik terhadap konten materi yang dibawa, sehingga tidak dapat dipakai untuk konten materi lainnya. \textcite{froggy2021media} juga termasuk pada kategori ini, karena implementasinya hanya spesifik pada materi yang dibawa, khususnya pada materi pengembangan web.

Metode non-visual kebanyakan menggunakan Web IDE dalam pembelajarannya. Pengguna dapat berinteraksi dengan Web IDE sehingga mereka dapat melatih kemampuan pemecahan masalah dan pemrograman praktis. Web IDE yang digunakan biasanya memiliki kapabilitas untuk mengembalikan hasil keluaran dari eksekusi kode, serta melakukan penilaian terhadap kebenaran implementasi. Web IDE juga dapat menerima berbagai macam bahasa sesuai dengan kebutuhan materi yang sedang dipelajari. Selain itu, terdapat juga Web IDE yang menyimulasikan terminal ketimbang editor kode seperti yang dapat dilihat di Katacoda \parencite{katacoda2021media}. Terdapat juga platform seperti \textcite{progate2021media} yang memanfaatkan pembelajaran interaktif dengan pendekatan "melalui" penyampaian materi secara visual dan naratif. Namun, kebanyakan dari implementasi metode ini hanya berfokus pada aspek latihan pemrograman praktis saja, tidak memerhatikan permasalahan terhadap pelajar yang telah dibahas sebelumnya.

Diluar dari platform-platform kelas pembelajaran pemrograman, sebenarnya terdapat berbagai macam pembelajaran interaktif yang menawarkan solusi yang lebih variatif, seperti pada \textcite{tran2013interactive} IDE yang dibuat digabungkan dengan aspek interaktif berupa kolaborasi antara pelajar dengan pelajar lainnya. Terdapat juga implementasi pada \textcite{guo2013pythontutor} serta \textcite{moons2013pilot} yang menerapkan pendekatan keempat berupa visualisasi eksekusi kode. Visualisasi eksekusi kode menjadi salah satu aspek yang menarik karena tidak banyak ditemukan implementasinya pada platform kelas pembelajaran pemrograman yang sudah ada.

% Selain itu, Web IDE bisa dikembangkan lebih lanjut pada aspek kolaboratif \parencite{tran2013interactive} ataupun mendukung visualisasi langkah eksekusi program seperti pada \textcite{guo2013pythontutor}. Visualisasi langkah eksekusi menjadi salah satu aspek yang menarik karena tidak banyak ditemukan implementasinya pada platform pembelajaran pemrograman yang sudah terkenal.

% REVISI II
% KodeBareng adalah platform berbasis web sebagai sarana pembelajaran pemrograman menggunakan gamifikasi. Metode pembelajarannya berbasis teks, visual interaktif, serta kuis-kuis. Kuis-kuis tersebut dapat berupa pilihan ganda, isian singkat, ataupun pencocokan jawaban yang membahas materi terkait yang sudah dipelajari. Dengan adanya kuis tersebut, diharapkan pelajar tidak hanya bisa memahami teori namun juga implementasi kode dari teori yang telah dijabarkan, seperti dengan adanya pertanyaan terkait hasil keluaran suatu kode. Namun, pertanyaan kuis semacam itu memiliki limitasi karena pelajar tidak dapat membuat sendiri implementasi kode sehingga susah untuk mengukur pemahaman praktis dan kemampuan pemecahan masalah menggunakan teori yang telah diberikan. Padahal, pemrograman adalah suatu kemampuan yang membutuhkan banyak latihan praktik implementasi agar pemahaman dapat tercapai secara optimal.

% Alternatif lain juga bisa dengan memberikan permasalahan yang harus dipecahkan kemudian pelajar dapat mengimplementasikan sendiri terlebih dahulu di mesinnya lalu melakukan submisi kode pada platform. Hal ini menimbulkan permasalahan lain seperti susahnya melakukan pemecahan permasalahan pada instalasi spesifik mesin yang digunakan oleh pelajar, adanya perbedaan versi dari lingkungan mesin pelajar dan mesin penguji yang dapat menimbulkan masalah dalam pengetesan, dan juga membuat platform lebih terbatas portabilitasnya karena instalasi hanya terbatas pada mesin dan lingkungan sistem tertentu. Permasalahan lain juga timbul dari cara melakukan penilaian dan pengetesan kode yang telah dibuat, karena apabila dilakukan secara manual akan membutuhkan banyak tenaga kerja.

% REVISI I
% Saat ini, terdapat banyak sekali metode penyampaian materi pembelajaran pemrograman secara daring. Namun, agar dapat memvalidasi pengetahuan yang didapat diperlukan sistem pendukung yang dapat digunakan sebagai media praktik bagi para pelajar. Kurangnya praktik dan latihan dalam pembelajaran dapat membuat adanya pemisah antara pemahaman teori dan praktik. (\textit{!TODO: ceritakan paper mengenai ini})

% Untuk memvalidasi pengetahuan yang sudah dipelajari, terdapat banyak metode yang dapat digunakan. Salah satu metode yang sering dipakai adalah kuis yaitu serangkaian pertanyaan yang mengacu pada materi yang telah diberikan. Kuis dapat dibuat dalam berbagai macam bentuk seperti pilihan ganda, isian singkat, esai, dll. Metode esai dapat digunakan untuk memvalidasi logika dan pola pikir dari pemecahan masalah, namun karena esai tersebut merupakan kode maka harus terdapat sistem yang dapat mengeksekusi, menilai, serta memberikan \textit{feedback} dari hasil eksekusi kode tersebut layaknya pemrograman yang sebenarnya.

% Maka dari itu, diperlukan sistem pembelajaran pemrograman interaktif yang dapat dipakai pengguna untuk menuliskan kode, memberikan \textit{feedback} dari kode yang dibuat, serta menilai kebenaran dari kode tersebut. \textit{Feedback} yang diberikan berupa hasil eksekusi kode berupa pesan keluaran apabila kode berhasil dijalankan, serta pesan error apabila terjadi masalah dalam eksekusi kode. Pesan keluaran hasil eksekusi dapat dibandingkan dengan pesan keluaran yang seharusnya agar dapat dinilai dan diberitahukan kepada pengguna sehingga pengguna dapat mengetahui letak kesalahan dari kode yang diimplementasikan.

\section{Analisis Kebutuhan}
\blindtext

\begin{figure}[H]
  \centering
  \includegraphics[width=\textwidth]{chapter3/diagram_usecase_v2.jpg}
  \caption{Use Case ILE KodeBareng} \label{fig:diagram-usecase}
\end{figure}

\blindtext

\begin{longtable}[c]{|l|>{\setlength{\baselineskip}{0.75\baselineskip}}p{0.5\linewidth}|>{\setlength{\baselineskip}{0.75\baselineskip}}p{0.3\linewidth}|}
  \caption{Use Case ILE KodeBareng}
  \label{tab:usecase}                                                       \\
  \hline
  \rowcolor{gray!30}
  \textbf{ID} & \textbf{Kebutuhan}                    & \textbf{Penjelasan} \\ \hline
  \endfirsthead
  %
  \endhead
  %
  UC-01       & Melihat materi pembelajaran           &                     \\ \hline
  UC-02       & Mengerjakan soal pemrograman          &                     \\ \hline
  UC-03       & Melihat visualisasi eksekusi kode     &                     \\ \hline
  UC-04       & Mengetahui hasil penilaian kode       &                     \\ \hline
  UC-05       & Memberikan masukkan (input)           &                     \\ \hline
  UC-06       & Mengetahui hint kesalahan pada kode   &                     \\ \hline
  UC-07       & Menjalankan kode                      &                     \\ \hline
  UC-08       & Menolak kode pengguna                 &                     \\ \hline
  UC-09       & Deteksi blacklisted commands          &                     \\ \hline
  UC-11       & Menilai kode                          &                     \\ \hline
  UC-12       & Mengolah stack trace eksekusi program &                     \\ \hline
  UC-13       & Deteksi kebutuhan input               &                     \\ \hline
  UC-14       & Deteksi error                         &                     \\ \hline
  UC-15       & Menentukan blacklisted commands       &                     \\ \hline
  UC-16       & Memasukkan materi pembelajaran        &                     \\ \hline
  UC-17       & Membuat soal pemrograman              &                     \\ \hline
  UC-18       & Memasukkan test case soal             &                     \\ \hline
  UC-19       & Memasukkan hint kesalahan kode        &                     \\ \hline
\end{longtable}

[\hl{TODO: TAMBAHKAN USE CASE DIAGRAMS, DESKRIPSI PROGRAM DIMASUKKAN KESINI, LALU KASIH TABEL KETERKAITAN USECASE DAN REQUIREMENT}]

\begin{longtable}[c]{|l|>{\setlength{\baselineskip}{0.75\baselineskip}}p{0.5\linewidth}|>{\setlength{\baselineskip}{0.75\baselineskip}}p{0.3\linewidth}|}
  \caption{Kebutuhan fungsional sistem}
  \label{tab:fungsional}                                                                                                                                                                                                                                                  \\
  \hline
  \rowcolor{gray!30}
  \textbf{ID} & \textbf{Kebutuhan}                                                                       & \textbf{Penjelasan}                                                                                                                                            \\ \hline
  \endfirsthead
  %
  \endhead
  %
  KB-F-01     & Pengguna memasukkan kode program pada sistem                                             & Menggunakan kakas Monaco Editor                                                                                                                                \\ \hline
  KB-F-02     & Pengguna memasukkan input program pada sistem                                            & Masukan diminta apabila kode membutuhkan masukan                                                                                                               \\ \hline
  KB-F-03     & Pengguna dapat menjalankan kode                                                          & Pengguna dapat melihat hasil eksekusi kode                                                                                                                     \\ \hline
  KB-F-04     & Pengguna dapat melihat hasil visualisasi eksekusi kode                                   & Pengguna dapat melihat perubahan data dan \textit{flow} program yang dijalankan                                                                                \\ \hline
  KB-F-05     & Pengguna mendapat penilaian dari hasil eksekusi kode                                     & Penilaian berdasarkan teknik \textit{blackbox autograding}                                                                                                     \\ \hline
  KB-F-06     & Pengguna mendapat \textit{hint} kesalahan pada kode program                              & -                                                                                                                                                              \\ \hline
  KB-F-07     & Pengguna dapat melihat pesan error pada program                                          & -                                                                                                                                                              \\ \hline
  KB-F-08     & Sistem menolak kode pengguna apabila terdapat perintah-perintah yang tidak diperbolehkan & Keterbatasan eksekusi dapat dilihat disini (TBA)                                                                                                               \\ \hline
  KB-F-09     & Sistem menilai kode program menggunakan \textit{test case}                               & -                                                                                                                                                              \\ \hline
  KB-F-10     & Sistem mengolah \textit{stack trace} eksekusi program                                    & \textit{Stack trace} berisi daftar fungsi, variabel, modul, serta \textit{stack frame} pada tiap langkah eksekusi program. Debugger yang digunakan adalah pdb. \\ \hline
  KB-F-11     & Administrator menentukan perintah-perintah yang tidak diperbolehkan                      & -                                                                                                                                                              \\ \hline
  KB-F-12     & Administrator memasukkan konten materi pembelajaran                                      & Dapat memasukkan visualisasi kode pada materi pembelajaran                                                                                                     \\ \hline
  KB-F-13     & Administrator membuat soal pemrograman                                                   & -                                                                                                                                                              \\ \hline
  KB-F-14     & Administrator memasukkan \textit{test case} soal pemrograman                             & \textit{Test case} tiap soal bisa lebih dari satu dan dapat memiliki input output yang berbeda-beda                                                            \\ \hline
  KB-F-15     & Administrator memasukkan \textit{hint} kesalahan yang dapat ditampilkan pada soal        & -                                                                                                                                                              \\ \hline
\end{longtable}

\blindtext

% Please add the following required packages to your document preamble:
% \usepackage{longtable}
% Note: It may be necessary to compile the document several times to get a multi-page table to line up properly
[\hl{TODO: Pilih 3 aja, performance sama security udah fix}]
\begin{longtable}[c]{|l|l|>{\setlength{\baselineskip}{0.75\baselineskip}}p{0.6\linewidth}|}
  \caption{Kebutuhan non-fungsional sistem}
  \label{tab:non-fungsional}                                                                                       \\
  \hline
  \rowcolor{gray!30}
  \textbf{ID} & \textbf{Parameter} & \textbf{Kebutuhan}                                                            \\ \hline
  \endfirsthead
  %
  \endhead
  %
              & Performance        & Sistem mengembalikan hasil eksekusi paling lama 10 detik                      \\ \hline
              & Reliability        & Sistem menangani 1000 pengguna yang mengerjakan latihan soal secara bersamaan \\ \hline
              & Maintainability    & Implementasi kode sistem menggunakan praktis \textit{clean code}              \\ \hline
              & Testability        & Sistem dirancang sedemikian rupa sehingga mudah dilakukan pengetesan          \\ \hline
              & Security           & Sistem tidak membiarkan terjadinya eksekusi kode berbahaya                    \\ \hline
\end{longtable}
\blindtext

\begin{longtable}[c]{|l|>{\setlength{\baselineskip}{0.75\baselineskip}}p{0.5\linewidth}|}
  \caption{Keterkaitan SRS dan Use Case}
  \label{tab:srs-usecase}                \\
  \hline
  \rowcolor{gray!30}
  \textbf{ID SRS} & \textbf{ID Use Case} \\ \hline
  \endfirsthead
  %
  \endhead
  %
  KB-F-01         & UC-02                \\ \hline
  KB-F-02         & UC-05, UC-13         \\ \hline
  KB-F-03         & UC-02, UC-04, UC-07  \\ \hline
  KB-F-04         & UC-01, UC-02, UC-03  \\ \hline
  KB-F-05         & UC-04, UC-11         \\ \hline
  KB-F-06         & UC-06                \\ \hline
  KB-F-07         & UC-02, UC-14         \\ \hline
  KB-F-08         & UC-08, UC-09         \\ \hline
  KB-F-09         & UC-11                \\ \hline
  KB-F-10         & UC-12                \\ \hline
  KB-F-11         & UC-15                \\ \hline
  KB-F-12         & UC-16                \\ \hline
  KB-F-13         & UC-17                \\ \hline
  KB-F-14         & UC-18                \\ \hline
  KB-F-15         & UC-19                \\ \hline
  KB-NF-01        &                      \\ \hline
  KB-NF-02        &                      \\ \hline
  KB-NF-03        &                      \\ \hline
\end{longtable}

\section{Rancangan Solusi}
 [\hl{TODO: JELASIN LAGI INTERAKTIVITASNYA DIMANA, JELASIN FLOW PROGRAMNYA SEPERTI APA}]

Dalam tugas akhir ini, dibuat prototipe ILE yang terdiri dari beberapa komponen yaitu Web IDE pada \textit{frontend} dan sistem eksekutor kode pada \textit{backend} yang terpisah dari \textit{backend} KodeBareng.

\begin{figure}[H]
  \centering
  \includegraphics[width=0.5\textwidth]{chapter3/diagram_blok.png}
  \caption{Rancangan Solusi ILE} \label{fig:diagram-blok}
\end{figure}

Keterangan gambar:
\begin{itemize}
  \setlength\itemsep{-0.2cm}
  \item Implementasi: Komponen diimplementasikan dari awal hingga akhir.
  \item Modifikasi: Komponen hanya perlu diubah beberapa bagian.
  \item Bawaan: Komponen tidak berubah.
\end{itemize}

Seperti yang dapat dilihat pada \autoref{fig:diagram-blok} sebelumnya, pada \textit{frontend} akan dibuat komponen Web IDE berupa editor kode yang dapat menerima masukan kode serta memberikan \textit{basic syntax highlighting}. Web IDE dapat menampilkan hasil keluaran ataupun kesalahan eksekusi yang dikembalikan dari \textit{backend}, serta hasil penilaian dari pengetesan kode. Web IDE juga dapat menampilkan visualisasi langkah jalannya program agar dapat memudahkan proses pembelajaran dan pencarian letak masalah dalam implementasi kode.

Pada \textit{backend}, akan dibuat sistem eksekutor kode yang terhubung ke backend KodeBareng. Sistem ini menerima kode dan informasi lainnya dari \textit{backend} lalu melakukan eksekusi kode yang akan dinilai berdasarkan hasil keluarannya. Sistem ini juga dapat terhubung pada \textit{backend} KodeBareng apabila dibutuhkan integrasi lain di kemudian hari.

\subsection{Rancangan Modul}
\autoref{fig:diagram-komponen} Berikut adalah diagram komponen dari sistem yang akan dibuat (TODO: Kalau yang margun udah jadi ntar musti di highlight yg dikerjain yg mana)

\begin{figure}[H]
  \centering
  \includegraphics[width=\textwidth]{chapter3/diagram_komponen.png}
  \caption{Diagram komponen} \label{fig:diagram-komponen}
\end{figure}
\blindtext
