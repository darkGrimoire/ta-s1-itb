\chapter{Rencana Penyelesaian Masalah}

Tujuan utama penulisan bab ini adalah untuk menguraikan rencana penyelesaian masalah tugas akhir yang akan dieksekusi secara utuh pada saat pelaksanaan Tugas Akhir II. Bab ini merupakan bab penutup Laporan Tugas Akhir I yang dapat dipandang sebagai bab yang akan menjembatani perpindahan ke proses pelaksanaan Tugas Akhir II. Pengembangan lebih lanjut dari bab ini dapat menjadi bagian dari bab Deskripsi Solusi pada Laporan Tugas Akhir.

\textbf{Kerangka penyelesaian masalah:}
\begin{enumerate}
  \item Analisis masalah apa saja yang ditemukan oleh pemula yang ingin mempelajari pemrograman secara daring. \\
        Apa saja hambatannya? Apa yang kurang dari proses pembelajaran materi saja? Cari tahu tentang adanya indikasi dari kurangnya latihan secara praktik sehingga menyebabkan terjadinya diskrepansi antara materi dan praktik
  \item Analisis solusi yang dibutuhkan \\
        Infrastruktur dari solusinya disesuaikan dengan hasil riset terkait infrastruktur autograder dengan performa dan keamanan yang baik, serta penggunaan Websocket untuk komunikasi client-server.
  \item Rancangan Solusi \\
        Rancangan solusi yang akan diimplementasikan. Dibuat komponen infrastrukturnya dan komunikasi antara komponen dengan sistem yang sudah ada pada kodebareng dan disesuaikan dengan tujuan performa yang diinginkan oleh kodebareng.
\end{enumerate}